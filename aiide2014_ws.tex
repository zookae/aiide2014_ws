\documentclass[10pt,a4paper]{article}
\usepackage[utf8]{inputenc}
\usepackage{amsmath}
\usepackage{amsfonts}
\usepackage{amssymb}

\usepackage{caption}


\newcommand{\mytodo}[1]{\textbf{[[#1]]}}

\author{Alexander Zook\footnote{primary contact} \hspace{1pt} and Michael Cook}
\title{Experimental AI for Games Workshop}
\date{}




\begin{document}

\maketitle

\section{Workshop Goals}

The primary goal of the ``Experimental AI for Games'' workshop is to foster innovation in how AI is used for games as a whole.
This workshop is not a continuation of a previous workshop.
Four key themes are related to this goal:
\begin{enumerate}
\item Fostering research into new AI systems in games (e.g. AIIDE 2013's presentation on \textit{Third Eye Crime} that visualized AI logic to create gameplay)
\item Increased cross-pollination from areas of AI not used in games (e.g. using computer vision for augmented reality games or applying natural language processing and generation technologies for dialog-based interactions in games)
\item Encouraging new kinds of games and interactive entertainment based on traditional AI methods (e.g. the use of drama management in \textit{Left 4 Dead} or neural net learning in \textit{Black \& White})
\item More broadly improving game development and design practices via AI (e.g. procedural content generation methods, intelligent design tools for analysis, or methods to automate simple design tasks like playtesting)
\end{enumerate}

Ultimately the workshop aims to engage the game AI community in building games to better address these topics.
The workshop focus will be on discovering and discussing problems typically overlooked by existing paradigms for using AI in games.
Creating more games based on AI methods and using existing tools to support game development can yield more playable experiences.
Further, engaging in concrete game design and development processes can reveal new problems for AI in interactive entertainment that may become full research topics for the broader AIIDE community.

Topics of interest to the workshop include (but are not limited to): 
\begin{itemize}
\itemsep0em
\item applications of natural language processing and generation as game mechanics
\item uses of computer vision in games
\item intelligent interactive narrative
\item novel social game agents
\item pathplanning algorithms used in non-physical spaces
\item procedural content generation in game development or as a game mechanic
\item AI game design tools
\item formal / computational models of game design and aesthetics
\item automated game generation
\end{itemize}

\section{Preliminary Format and Schedule}

We plan to hold a one day workshop based on proposing and discussing experimental applications of AI in games.
The workshop will solicit submissions as position and/or prototype/work-in-progress papers for experimental applications of AI in games or novel AI-based games.
Submissions will be 4-6 pages long, with one additional page allow for only acknowledgments and references.
Papers will focus on argumentation (e.g. using thought experiments, clear reasoning, a prototype, and so on) for a new application of AI to games.
We will encourage a workshop atmosphere by focusing on presenting and discussing ideas over validating techniques through empirical evaluation. 
The workshop will emphasize on novel problems, qualitative expectations, thought experiments, and initial prototypes.

During the workshop authors will briefly present an experimental application of AI in games with the emphasis of the workshop being discussion on the topics presented.
We will hold breakout discussions around the key topics addressed by papers.
Further, we plan to hold invited talks to encourage consideration of related areas and known development challenges.
We will invite speakers from related areas (e.g. musical metacreation researchers or developers of design analysis or content creation tools) to spur discussion on applications of related technologies.
Local developers and designers will also be invited to talk about their experiments with AI in games.

We also plan to hold a continuation of the AI and Game Aesthetics Workshop's DAGGER event.
The event will encourage workshop participants and local developers to demonstrate their games that use AI in some way and encourage discussion about making other games.

%\begin{table}[tb]
%\centering
%\caption*{\textbf{Planned Schedule}}
%\begin{tabular}{c}
%\hline
%Introduction and Welcome \\
%Keynote                  \\
%Paper Session 1          \\
%Lunch                    \\
%Paper Session 2          \\
%Invited Talks            \\
%DAGGER event             \\ \hline
%\end{tabular}
%\end{table}


\section{Tentative Participants}
We expect participants from AIIDE 2013's AI and Game Aesthetics and AI in the Game Design Process to be interested as the topic matter and workshop format are similar.
Those interested in the research AI game jam (RAIGJ) are also likely to be interested in this workshop due to the overlapping interests and topic matter.

Tentative attendees include: Eric Butler, Jonathan Tremblay, Kristin Siu, Antonios Liapis, ...

\subsection*{Tentative Program Committee}
\begin{itemize}
\itemsep0em
\item Simon Colton (Goldsmiths)
\item (?) Mark Riedl (Georgia Tech)
\item Gillian Smith (Northwestern)
\item Chong-U Lim (MIT)
\item Tom Betts (Big Robot Games)
\item Pippin Barr (Malta)
\item Antonios Liapis (ITU Copenhagen)
\item (?) Kate Compton
\item 
\end{itemize}


\section{Organizing Committee}
\begin{itemize}
\itemsep0em
\item Alexander Zook, Ph.D. student in the School of Interactive Computing at the Georgia Institute of Technology
\begin{itemize}
\itemsep0em
\item a.zook@gatech.edu
\item 1-773-758-8191
\item 6303 Renaissance Way NE
\item Atlanta, GA 30308
\end{itemize}
\item Michael Cook, Research Associate and Ph.D. Student in the Department of Computing at Goldsmiths College, University of London
\begin{itemize}
\itemsep0em
\item mike@gamesbyangelina.org
\item +44 7790 873388
\item Department of Computing, Goldsmiths College, University of London
\item Ben Pimlott Building, New Cross, London SE14 6NW, UK
\end{itemize}
\end{itemize}



\end{document}