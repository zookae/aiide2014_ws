\documentclass[10pt,a4paper]{article}
\usepackage[utf8]{inputenc}
\author{Alexander Zook and Michael Cook}
\title{Experimental AI in Games}
\date{}

\newcommand{\mytodo}[1]{\textbf{[[#1]]}}

\begin{document}

\maketitle

\section*{Abstract}



\section*{Report Body}

The Experimental AI in Games (EXAG) workshop fosters innovation in how AI is used in and for games.
Research in game AI has traditionally focused on intelligent adversarial agent behavior, exemplified by chess-playing systems such as Deep Blue.
Yet games as entertainment experiences involve far more than expert-level opponent play.
Game design and development introduce a variety of challenges in producing content, reasoning on how gameplay creates player experiences, and dynamically manipulating game worlds in response to fit players.
In-game systems also stand to benefit from advances in AI, including novel applications of computer vision, natural language processing, user modeling, and computational creativity to provide new kinds of input to and output from games or possibilities for gameplay experiences.
EXAG has two goals: encouraging long-term, blue sky thinking around new ways AI could be used in games and providing a venue for experiments on innovative uses of AI in games.


\mytodo{affiliations and full author lists?}
EXAG submissions this year clustered around \mytodo{three} themes: new game mechanics possible through AI, ways to gain knowledge for procedural content generation, and new roles for AI in game creation.
Game mechanics are traditionally a set of rules defined by game designers that govern the flow of game states.
Could AI systems provide different mechanics that were not previously possible?
Justus Robertson demonstrated how planning techniques can be used to enable on-line alterations of a game world in an interactive narrative to guide players along an intended story in a game world.
Ian Horswill showed how classical AI research in question answering and problem solving could yield innovative game mechanics based on mind control with open-ended game play.
Discussion around these systems considered the challenges inherent in creating AI-based game mechanics and what new avenues remained unexplored.


Procedural content generation systems create game content algorithmically.
Most systems, however, are concerned with creating low-level game assets and lack the semantic knowledge necessary to assemble content that is meaningful or sensible to humans.
Michael Cook presented two approaches to mining existing knowledge bases for information relevant to games.
`Google milking' harvests answers from feeding incomplete questions into a search engine.
These results can then be used to provide knowledge for game content including actions for in-game entities and relationships to inform in-game enemies or collectible content.
Alternatively, connecting elements from ConceptNet can be used to mine interesting relationships and serve as a source of inspiration for fictional game premises.
Mark Riedl presented the notion of `playing games to make games' in the context of gathering commonsense knowledge about the spatial relationships among objects for creating scenes in 3D games.
Participants discussed the many challenges of gathering knowledge: ensuring knowledge is relevant, handling potentially offensive information, and motivating contributors to provide information.


Game creation is a challenging task involving many questions around creating different types of content and ensuring the combination of content creates desired player experiences.
How can AI systems enhance the game creation process?
Gillian Smith presented several future roles for procedural content generation in games, including the notion of content generation to entertain a community of viewers.
Kazjon Grace took the notion of AI interfacing with game communities further by discussing the potential for computational creativity techniques to support game modding communities by working with the growing wealth of online game content resources.
Jonathan Tremblay addressed design support by instead considering how search algorithms can be used to automatically solve platformer game levels to provide feedback to level designers.
These papers stimulated discussions around how AI can better support designers and consider communities of both players and creators.


\mytodo{include RCAIIDE? mention demo session?}
RCAIIDE panel + discussion
demo session

The papers of the workshop were published as AAAI Press Technical Report WS-14-01.
\mytodo{update to right info}


\section*{Authors}
\begin{enumerate}
\item Alexander Zook (a.zook@gatech.edu) is a PhD Candidate in the School of Interactive Computing at the Georgia Institute of Technology.
\item Michael Cook (mike@gamesbyangelina.org) is a Research Associate and Ph.D. Student in the Department of Computing at Goldsmiths College, University of London.
\end{enumerate}

\end{document}