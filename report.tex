\documentclass[10pt,a4paper]{article}
\usepackage[utf8]{inputenc}
\author{Alexander Zook and Michael Cook}
\title{Experimental AI in Games}
\date{}

\newcommand{\mytodo}[1]{\textbf{[[#1]]}}

\begin{document}

\maketitle

\section*{Abstract}
The Experimental AI in Games (EXAG) workshop fosters innovation in how AI is used in and for games.
EXAG has two goals: encouraging long-term, blue sky thinking around new ways AI could be used in games and providing a venue for experiments on innovative uses of AI in games.
Submissions clustered around three themes: new game mechanics possible through AI, ways to acquire game design knowledge, and new roles for AI in game creation.



\section*{Report Body}

The Experimental AI in Games (EXAG) workshop fosters innovation in how AI is used in and for games.
Research in game AI has traditionally focused on intelligent adversarial agent behavior, exemplified by chess-playing agents like Deep Blue.
Yet games as entertainment experiences involve far more than expert-level opponent play.
Game design and development include a plethora of challenges in producing content and reasoning on how gameplay creates player experiences.
AI techniques can help address these challenges.
Further, many AI areas---including computer vision, natural language processing, user modeling, and computational creativity---have the potential unlock new possibilities for gameplay experiences.
EXAG has two goals: encouraging long-term, blue sky thinking around new ways AI could be used in games and providing a venue for experiments on innovative uses of AI in games.


EXAG submissions this year clustered around three themes: new game mechanics possible through AI, ways to acquire game design knowledge, and new roles for AI in game creation.
Game mechanics are rules that govern the flow of game states.
Could AI systems enable games using mechanics that were previously impossible?
Justus Robertson (North Carolina State University) demonstrated how planning techniques can be used to enable on-line alterations of a game world in an interactive narrative to guide players along an intended story in a game world.
Ian Horswill (Northwestern University) showed how classical AI research in question answering and problem solving could yield innovative game mechanics based on mind control with natural language input.
Participants discussed the challenges in creating AI-based game mechanics and potential avenues for AI-based games.


Procedural content generation systems create game content algorithmically.
Most systems, however, are concerned with creating game assets and lack the semantic knowledge necessary to assemble content that is meaningful and sensible to humans.
Michael Cook (Goldsmiths College, University of London) presented two approaches to mining existing knowledge bases for information relevant to games.
`Google milking' harvests answers from feeding incomplete questions into a search engine.
These results provide knowledge for game content including actions for in-game entities and relationships between entities to suggest in-game enemies or collectible content.
Alternatively, chaining relations among concepts in ConceptNet can be used to find inspirational relationships for fictional game premises.
Mark Riedl (Georgia Institute of Technology) presented the notion of `playing games to make games' as a way to gather game-relevant knowlege.
This approach supported the automated generation of scenes in 3D games from text-based descriptions by using a game to collect relevant spatial relationship data.
Discussions concerned the challenges of gathering knowledge: ensuring knowledge is relevant, handling potentially offensive or inappropriate information, and motivating contributors to knowledge corpora.


Game creation is challenging: designers make game content to indirectly create player experiences.
How can AI systems enhance the game creation process?
Gillian Smith (Northeastern University) presented several future roles for procedural content generation in games, including the notion of generating content that entertains a community of viewers.
Kazjon Grace (University of North Carolina at Charlotte) took the notion of AI interfacing with game communities further by discussing the potential for computational creativity techniques to support game modding communities by working with the growing wealth of online game content resources.
Jonathan Tremblay (McGill University) addressed design support by comparing how different search algorithms automatically solve game levels as a source of feedback to level designers.
These papers stimulated discussion around how AI can improve the experience of game viewers and support game creators.


EXAG also included a panel discussion from the Research in Cognitive-based Approaches to Intelligent Interactive Digital Entertainment workshop.
Participants discussed the potential for new cognitive interfaces to provide alternative input to games and the possible uses of games for investigating human (and dog) cognition.
EXAG concluded with a demonstration session where Justus Robertson showed his system guiding players along a story in both a text-based game and a 2D adventure game.
Tommy Thompson (University of Derby) also demonstrated an approach to automated pathfinding to enable new forms of interaction in touch-based games and analysis of player facility with different control methods.


EXAG was chaired by Alexander Zook and Michael Cook; the papers of the workshop were published as AAAI Press Technical Report WS-14-16.



\section*{Authors}
\begin{enumerate}
\item Alexander Zook (a.zook@gatech.edu) is a PhD Candidate in the School of Interactive Computing at the Georgia Institute of Technology.
\item Michael Cook (mike@gamesbyangelina.org) is a Research Associate and Ph.D. Student in the Department of Computing at Goldsmiths College, University of London.
\end{enumerate}

\end{document}